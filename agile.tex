\subsection{Méthodologie Agile}\label{subsec:agile}

L'équipe de développement de la géolocalisation de SuiviDeFlotte -- comme les autres équipes de développement de l'entreprise -- travaille selon la méthodologie agile SCRUM. Cette méthodologie est une approche de gestion de projet qui met l'accent sur la flexibilité, la collaboration et la livraison continue. Elle est largement utilisée dans le développement de logiciels et peut également être appliquée à d'autres domaines. SCRUM divise un projet en cycles appelés ``itérations'' ou ``sprints'' de courte durée, généralement de deux à quatre semaines, pendant lesquels une partie du travail est accomplie et livrée.

Les termes clés dans la méthodologie SCRUM comprennent :

\begin{description}
    \item[Product Owner (Propriétaire du Produit)] La personne responsable de définir et de prioriser les éléments du produit à développer. Le Propriétaire du Produit représente les besoins des utilisateurs et des parties prenantes.
    \item[Scrum Master (Maître de Scrum)] Le facilitateur du processus SCRUM. Le Scrum Master s'assure que l'équipe suit les principes SCRUM, élimine les obstacles et favorise un environnement de travail efficace.
    \item[Équipe de Développement] Le groupe de professionnels chargé de concevoir, développer, tester et livrer les éléments du produit à la fin de chaque sprint.
    \item[Product Backlog (Carnet de Produit)] Une liste dynamique et priorisée de toutes les fonctionnalités, tâches et améliorations potentielles du produit. Elle est gérée par le Propriétaire du Produit.
    \item[Sprint Planning (Planification de Sprint)] Une réunion au début de chaque sprint au cours de laquelle l'équipe choisit les éléments du Carnet de Produit à inclure dans le sprint et planifie comment les livrer.
    \item[Sprint Backlog (Carnet de Sprint)] La liste des éléments du Carnet de Produit choisis pour le sprint en cours, ainsi que les tâches nécessaires à leur réalisation.
    \item[Daily Scrum (Mêlée Quotidienne)] Une courte réunion quotidienne pendant laquelle l'équipe de développement partage les progrès réalisés depuis la dernière réunion, discute des tâches à accomplir et identifie les obstacles potentiels.
    \item[Sprint Review (Revue de Sprint)] Une réunion à la fin de chaque sprint au cours de laquelle l'équipe présente les éléments terminés et recueille les commentaires du Propriétaire du Produit et des parties prenantes.
    \item[Sprint Retrospective (Rétrospective de Sprint)] Une réunion après la Revue de Sprint au cours de laquelle l'équipe examine le déroulement du sprint et identifie les points forts et les points à améliorer.
    \item[User Story (Histoire Utilisateur)] Une Histoire Utilisateur est une courte description d'une fonctionnalité ou d'un aspect du produit, racontée du point de vue de l'utilisateur. Elle suit généralement le format ``En tant que [utilisateur], je veux [action] afin de [objectif]''. Les Histoires Utilisateurs sont des éléments du Carnet de Produit et aident à définir les fonctionnalités du produit du point de vue de l'utilisateur.
    \item[Story Point (Point d'Histoire)] Le Point d'Histoire est une unité relative utilisée pour estimer la complexité, l'effort et la taille des Histoires Utilisateurs ou des tâches de développement. Il n'a pas de valeur absolue, mais il sert à comparer la difficulté relative entre différentes Histoires Utilisateurs. Les équipes de développement attribuent des points d'histoire lors des estimations, ce qui les aide à planifier la quantité de travail qu'elles peuvent accomplir dans un sprint donné.
    \item[Epic (Épique)] Un Épic est une unité de travail plus large que les Histoires Utilisateurs individuelles. Il représente généralement un ensemble de fonctionnalités, de tâches ou de travaux qui sont trop importants pour être traités dans un seul sprint. Les Épics sont souvent des objectifs à long terme qui sont décomposés en Histoires Utilisateurs plus petites et gérables. Ils aident à organiser et à structurer le développement du produit en regroupant des éléments liés autour d'un thème ou d'un objectif commun. Les Épics sont inclus dans le Carnet de Produit et sont priorisés en fonction de leur valeur pour l'utilisateur et du contexte global du projet.
\end{description}

SCRUM encourage la transparence, l'adaptabilité et la collaboration continue entre les membres de l'équipe et les parties prenantes, ce qui permet de s'adapter aux changements et de fournir rapidement de la valeur tout au long du projet.

Dans l'équipe Géoloc, le processus suit un calendrier de sprints de deux semaines. Chaque cycle commence par un ensemble de réunions clés qui ont lieu tous les deuxièmes mardis. Cette journée englobe la Revue de Sprint, la Rétrospective de Sprint et la Planification de Sprint. Lors de la Revue de Sprint, les éléments achevés sont présentés au Propriétaire du Produit et aux parties prenantes, les retours sont recueillis et les priorités sont ajustées si nécessaire. La Rétrospective de Sprint offre l'opportunité à l'équipe de réfléchir aux succès et d'identifier les domaines à améliorer, favorisant une culture d'amélioration continue. Ensuite, la Planification de Sprint implique la sélection des Histoires Utilisateurs à inclure dans le sprint à venir, en tenant compte de leur complexité et de leur priorité.

Chaque jour ouvrable débute par une Mêlée Quotidienne de 15 minutes, au cours de laquelle les progrès depuis la dernière réunion sont passés en revue, les tâches sont discutées et les obstacles potentiels sont identifiés. Ces réunions se déroulent généralement en ligne via Google Meet pour accueillir les collègues travaillant à distance et garantir la participation de tous les membres de l'équipe, peu importe leur emplacement.

Actuellement, les rôles du Propriétaire du Produit et du Maître de Scrum ont été temporairement fusionnés en une seule personne. Cette configuration permet au Propriétaire du Produit de gérer les responsabilités du projet tout en facilitant le processus SCRUM, en maintenant une communication fluide avec l'équipe de développement.

Pour l'estimation de la complexité des tâches, des Points d'Histoire basés sur la séquence de Fibonacci (1, 2, 3, 5, 8, 13) sont utilisés. Cette approche aide à attribuer des valeurs relatives à différentes tâches et assure une évaluation cohérente de la charge de travail.

Enfin, pour maintenir la communication et la cohésion de l'équipe, les réunions en personne sont maintenues pour les mardis tous les deuxièmes semaines. Cette pratique encourage les échanges directs et la collaboration étroite entre les membres de l'équipe. Cette approche garantit l'alignement et facilite la résolution rapide de tout problème ou obstacle pouvant survenir.