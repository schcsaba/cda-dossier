\subsection{Le processus d'importation d'un fichier}

Dans cette sous-section, nous suivrons le destin d'un fichier pendant son importation, depuis son arrivée à l'API dans le corps d'une requête POST, jusqu'à l'enregistrement de ses données dans la table appropriée de la base de données adéquate. Nous utiliserons à titre d'exemple un fichier LCCC (La Compagnie des Cartes Carburant SAS) qui est un type de fichier de transactions de carburant au format \Verb|csv| (Figure~\ref{fig:lccc}).

\begin{figure}[ht]
    \centering
    \includegraphics[width=\textwidth]{img/lccc-csv}
    \caption{Exemple d'un fichier LCCC.}
    \label{fig:lccc}
\end{figure}

Pour l'API, il est possible d'envoyer des fichiers dans le corps d'une requête POST au format encodé en base64, en tant que partie d'un objet JSON bien défini, comme démontré dans le Code source~\ref{code:lccc-payload}. Cette charge utile est validée avant d'être acceptée par l'API, plus précisément l'application Laravel utilise la méthode \Verb|rules()| de la classe \Verb|UploadStoreRequest| créée par nous pour la valider (Code source~\ref{code:upload-store-request}).

Cette classe se trouve dans le dossier \Verb|app/Http/Requests|. Elle étend la classe \Verb|FormRequest| de Laravel, permettant ainsi de définir les règles de validation pour la requête de stockage (store request) d'import de fichiers.

La méthode \Verb|rules()| définit les règles de validation pour les différentes parties de la requête. Les règles sont définies sous forme d'un tableau associatif. Parmi les règles définies :

\begin{itemize}
    \item Le champ \Verb|client_app| doit être présent, être un nombre entier et être une des valeurs autorisées spécifiées dans la configuration du projet (\Verb|allowed_client_app|).
    \item Le champ \Verb|parameters| doit être un tableau avec au moins deux éléments.
    \item Les éléments spécifiques du champ \Verb|parameters| (\Verb|client_id| et \Verb|user_id|) doivent être présents, être des nombres entiers et être obligatoires.
    \item Le champ \Verb|files| doit être un tableau avec au moins un élément.\footnote{Pour l'instant, l'API n'est pas en mesure de gérer plus d'un fichier. Si plusieurs fichiers sont présents dans la charge utile, ils ne sont pas pris en compte.}
    \item Les éléments spécifiques du champ \Verb|files| (\Verb|content|) doivent être présents, commencer par \Verb|data:text/plain;base64,| et avoir une longueur minimale de 27 caractères.
\end{itemize}

En somme, la classe \Verb|UploadStoreRequest| sert de mécanisme de validation pour les requêtes POST envoyées à l'API, en s'assurant que les données fournies sont conformes aux règles spécifiées, contribuant ainsi à la sécurité et à la cohérence des données entrantes.

La charge utile doit être envoyée à la route appropriée définie dans le fichier \Verb|routes/api.php| (Code source~\ref{code:file-import-route}).

\begin{code}
    \caption{La charge utile (payload) à envoyer contenant le fichier encodé au format base64 et certaines métadonnées associées.}
    \inputminted[samepage]{json}{code/lccc-payload.json}
    \label{code:lccc-payload}
\end{code}

\begin{code}
    \caption{La définition de la route d'importation de fichiers dans le fichier \Verb{routes/api.php}.}
    \inputminted[firstline=47,lastline=47]{php}{code/api.php}
    \label{code:file-import-route}
\end{code}

Cette route est définie pour une requête POST vers l'URL \Verb|/command/uploads/{category}/{type}|. Lorsqu'une requête POST est effectuée vers cette URL, elle sera traitée par la méthode \Verb|upload| du contrôleur \Verb|UploadController|. Cette méthode prendra en charge le traitement de la requête et la gestion de l'importation des fichiers correspondant à la catégorie et au type spécifiés dans l'URL. Dans le cas de notre fichier, la catégorie est \Verb|fuel-transactions|, le type est \Verb|lccc|. Cette définition de route établit une correspondance entre une URL POST spécifique et la méthode du contrôleur qui gère le traitement de la requête et les opérations nécessaires. Cela permet de créer une API RESTful qui répond aux requêtes POST adressées à cette URL particulière.

\begin{code}
    \caption{La classe \Verb{UploadStoreRequest}.}
    \inputminted{php}{code/UploadStoreRequest.php}
    \label{code:upload-store-request}
\end{code}

\subsubsection{La méthode \Verb{upload} de la classe \Verb{UploadController}}

La classe \Verb|UploadController| (Code source~\ref{code:upload-controller}) représente une classe de contrôleur dans le projet. Elle gère les requêtes en répondant à certaines routes définies dans l'API. Nous ne nous occupons ici que de la route d'importation des fichiers et de la méthode \Verb{upload} de la classe. Lorsqu'une requête POST est effectuée vers cette route, la méthode \Verb{upload} du contrôleur est appelée.

\begin{code}
    \caption{La version simplifiée de la classe \Verb{UploadController}.}
    \inputminted{php}{code/UploadController.php}
    \label{code:upload-controller}
\end{code}

La méthode \Verb{upload} prend trois paramètres : une instance de la classe \Verb{UploadStoreRequest} (Code source~\ref{code:upload-store-request}, page~\pageref{code:upload-store-request}) pour la validation de la requête, une chaîne \Verb{category} pour la catégorie du fichier et une chaîne \Verb{type} pour le type de fichier. La méthode utilise ensuite la classe \Verb{Artisan} pour appeler une commande Artisan personnalisée correspondant à l'importation du fichier en utilisant la catégorie et le type spécifiés. Les paramètres tels que le jeton d'autorisation, les données validées de la requête et les détails de la catégorie et du type sont transmis à la commande.

Après l'exécution de la commande, la sortie est récupérée et analysée en tant que JSON. Cette sortie représente le résultat de l'importation du fichier, qui est ensuite renvoyé sous forme de réponse JSON avec le code de statut 201 (Créé). En somme, la classe \Verb{UploadController} agit comme un pont entre les requêtes d'importation de fichiers, la validation des données et l'exécution de la commande associée pour traiter les fichiers importés et renvoyer les résultats appropriés.
\subsubsection{La classe abstraite \Verb{UploadTrigge} et ses enfants}

Dans l'API, les commandes Artisan personnalisées qui sont appelées par la méthode \Verb|upload| de la classe \Verb|UploadController| sont représentées par la classe abstraite \Verb|UploadTrigger| et ses enfants. Elles se trouvent dans le dossier \Verb|app/Console/Commands| du projet.

La classe \Verb|UploadTrigger| est une classe de commande, destinée à gérer le processus de déclenchement d'une importation de fichiers. Elle étend la classe \Verb|Command| de Laravel, fournissant ainsi la base pour définir des commandes spécifiques.

La classe contient des propriétés statiques \Verb|storageDirectory| et \Verb|workDirectory| pour les répertoires de stockage et de travail. Dans le constructeur (Code source~\ref{code:upload-trigger-constructor}), des arguments d'entrée (\Verb|input|, \Verb|token|, \Verb|category| et \Verb|type|) sont définis pour la commande, ainsi qu'une définition d'entrée pour les utiliser.

\begin{code}
    \caption{Le constructeur de la classe \Verb{UploadTrigger}.}
    \inputminted[firstline=28,lastline=38]{php}{code/UploadTrigger.php}
    \label{code:upload-trigger-constructor}
\end{code}

La méthode \Verb|handle()| (Code source~\ref{code:upload-trigger-handle}) est appelée lorsque la commande est exécutée. Elle enregistre un nouvel \Verb|Upload| en utilisant les paramètres fournis, puis déclenche un événement \Verb|StepStart| avec les paramètres nécessaires pour démarrer l'étape de traitement. Le statut de l'importation est mis à jour et les informations sont enregistrées dans un fichier journal.

\begin{code}
    \caption{La méthode \Verb{handle()} de la classe \Verb{UploadTrigger}.}
    \inputminted[firstline=40,lastline=62]{php}{code/UploadTrigger.php}
    \label{code:upload-trigger-handle}
\end{code}

La méthode \Verb|recordNewUpload()| (Code source~\ref{code:upload-trigger-record-new-upload}) enregistre les détails de l'importation dans la base de données, gère les fichiers associés et renvoie un objet \Verb|Upload|.

\begin{code}
    \caption{La méthode \Verb{recordNewUpload()} de la classe \Verb{UploadTrigger}.}
    \inputminted[firstline=65,lastline=88]{php}{code/UploadTrigger.php}
    \label{code:upload-trigger-record-new-upload}
\end{code}

La méthode \Verb|saveInDirectories()| (Code source~\ref{code:upload-trigger-save-in-directories}) gère l'enregistrement des fichiers dans les répertoires de stockage et de travail, tout en gérant les paramètres associés.

\begin{code}
    \caption{La méthode \Verb{saveInDirectories()} de la classe \Verb{UploadTrigger}.}
    \inputminted[firstline=90,lastline=121]{php}{code/UploadTrigger.php}
    \label{code:upload-trigger-save-in-directories}
\end{code}

L'objectif global de la classe est de gérer le flux de travail lié à l'importation de fichiers, de l'enregistrement initial à l'enregistrement des fichiers et à la mise à jour des statuts et des journaux correspondants. Cette classe de commande abstraite joue un rôle essentiel dans la gestion des processus d'importation de fichiers dans l'API, en encapsulant les étapes et les opérations associées dans des méthodes clés.

Cette classe contient les fonctionnalités communes des commandes d'importation de fichiers. Pour créer des commandes spécifiques à un fichier, il convient d'étendre cette classe et de définir les attributs spécifiques dans les classes enfants.

La classe \Verb|UploadTriggerLCCC| (Code source~\ref{code:upload-trigger-lccc}) est une classe de commande spécifique à notre fichier d'exemple, destinée à gérer le déclenchement de l'importation des transactions de carburant spécifiques à LCCC. Elle étend la classe abstraite \Verb|UploadTrigger|, héritant ainsi des fonctionnalités et du comportement définis dans la classe parente.

La propriété \Verb|signature| est définie pour cette commande, indiquant comment la commande doit être appelée à partir de la ligne de commande Laravel ou à partir du code, comme dans la méthode \Verb|upload| de la classe \Verb|UploadController|. Dans ce cas, la signature est \Verb|upload-fuel-transactions:lccc|.

La propriété \Verb|description| donne une brève description de ce que fait la commande. Dans ce cas, la description est \foreignquote{french}{Import fuel transactions LCCC}, ce qui indique clairement que la commande est utilisée pour l'importation des transactions de carburant spécifiques à LCCC.

\begin{code}
    \caption{La classe \Verb{UploadTriggerLCCC}.}
    \inputminted{php}{code/UploadTriggerLCCC.php}
    \label{code:upload-trigger-lccc}
\end{code}
\subsubsection{Les événements}

Comme nous l'avons vu, la méthode \Verb|handle| de la classe \Verb|UploadTrigger| déclenche l'événement appelé \Verb|NextStepEvent|.

Les événements dans Laravel sont un mécanisme qui permettent la communication entre différentes parties de l'application de manière souple et découplée. Ils facilitent la gestion des interactions entre les composants en permettant à une partie de déclencher un événement, tandis que d'autres parties (les écouteurs d'événements) peuvent réagir à cet événement de manière autonome. Les événements permettent de séparer les responsabilités et de rendre l'application plus modulaire, tout en facilitant l'ajout ou la modification de fonctionnalités sans perturber l'ensemble du système.

Le fonctionnement des événements dans Laravel repose sur trois éléments principaux :

\begin{enumerate}
    \item \textbf{Définition de l'événement :} Un événement est défini en tant que classe qui utilise le trait \Verb|Dispatchable|. Cette classe contient généralement des propriétés et des méthodes pour définir les informations spécifiques à l'événement.
    \item \textbf{Déclenchement de l'événement :} Une fois qu'on a créé la classe d'événement, on peut utiliser la méthode statique \Verb|dispatch()| sur la classe pour déclencher l'événement. Tous les arguments transmis à la méthode seront transmis au constructeur de l'événement. Cela notifie tous les écouteurs d'événements associés à cet événement.
    \item \textbf{Écouteurs d'événements :} Les écouteurs d'événements sont des classes qui réagissent aux événements spécifiques enregistrés. Ils écoutent les événements déclenchés et exécutent des actions prédéfinies en réponse à ces événements.
\end{enumerate}

Les abonnés d'événements (event subscribers) dans Laravel offrent une manière alternative d'organiser et de gérer les écouteurs d'événements. Au lieu d'attacher des écouteurs à des classes d'événements individuelles en utilisant la méthode \Verb|listen()|, les abonnés d'événements nous permettent de regrouper des écouteurs liés dans une seule classe. En raison de son caractère pratique, nous avons utilisé cette approche et nous avons créé l'abonné d'événement appelé \Verb|StepEventSubscriber|. Mais avant de l'examiner en détail, regardons la classe \Verb|NextStepEvent| (Code source~\ref{code:next-step-event}).

\begin{code}
    \caption{La classe \Verb{NextStepEvent}.}
    \inputminted{php}{code/NextStepEvent.php}
    \label{code:next-step-event}
\end{code}

La classe \Verb|NextStepEvent| utilise la trait \Verb|Dispatchable|. La classe a une propriété publique \Verb|step| qui est une instance de l'interface \Verb|StepInterface|. L'interface \Verb|StepInterface| définit les méthodes (\verb|start|, \Verb|process|, \Verb|getNameOfPreviousResultIn|, \Verb|getNameResultOut|) que doivent implémenter les classes qui représentent les maillons de téléchargement. Le constructeur de la classe prend une instance de l'interface \Verb|StepInterface| en argument et la stocke dans la propriété \Verb|step|.

Nous avons créé un autre événement en plus du \Verb|NextStepEvent|, le \Verb|StepErrorEvent|. Le code de cet événement est très similaire à celui de l'événement présenté ici. La seule différence réside dans le fait que le \Verb|StepErrorEvent| possède une propriété supplémentaire, le \Verb|message|. Cette propriété est une chaîne de caractères et elle est définie par le constructeur à la valeur de son deuxième argument. L'événement \Verb|StepErrorEvent| est déclenché par les maillons s'ils rencontrent une erreur au cours de leur exécution. Passons maintenant à l'abonné d'événement \Verb|StepEventSubscriber|.

La classe \Verb|StepEventSubscriber| implémente l'interface \Verb|ShouldQueue|, indiquant que les écouteurs doivent être placés dans une file d'attente pour un traitement différé. Les constantes \Verb|PROCESS_CONFIG_KEY| et \Verb|ERROR_CONFIG_KEY| sont définies pour représenter les clés de configuration utilisées dans les événements.

La méthode \Verb|handleNextStep| (Code source~\ref{code:step-event-subscriber-handle-next-step}) est l'écouteur pour l'événement \Verb|NextStepEvent|. Cet écouteur traite les maillons suivants lorsqu'un événement \Verb|NextStepEvent| est déclenché. Il récupère les paramètres nécessaires de l'événement et exécute les maillons de traitement définis dans la configuration. Tout d'abord, l'événement \Verb|NextStepEvent| fournit un objet \Verb|Step| qui représente le aillon actuel du processus. À partir de cet objet, l'événement peut extraire les paramètres nécessaires et l'emplacement du type d'importation (upload type). En utilisant le type d'importation, la méthode accède à la configuration de l'importation correspondante dans la classe \Verb|UploadType|.\footnote{Le code de la classe \Verb{UploadType} et la configuration de l'importation du fichier LCCC sont disponibles aux annexes (Code source~\ref{code:upload-type}, page~\pageref{code:upload-type}, Code source~\ref{code:lccc-config}, page~\pageref{code:lccc-config}).} Cette configuration contient des informations sur les maillons et les actions à effectuer pour ce type d'importation. Une fois que la configuration est trouvée, la méthode recherche le maillon suivant à exécuter. S'il y a un maillon suivant défini dans la configuration, la méthode extrait les informations sur ce prochain maillon, notamment sa classe et ses paramètres. Si un maillon suivant est trouvé, la méthode instancie ce maillon en utilisant la classe et les paramètres extraits. Ensuite, elle appelle la méthode \Verb|start| sur cette instance pour lancer l'exécution du maillon suivant. Si aucun maillon suivant n'est trouvé, cela signifie que l'importation est terminée. Dans ce cas, la méthode met à jour le statut de l'importation pour indiquer qu'elle est terminée.

\begin{code}
    \caption{La méthode \Verb{handleNextStep} de la classe \Verb{StepEventSubscriber}.}
    \inputminted[firstline=21,lastline=51]{php}{code/StepEventSubscriber.php}
    \label{code:step-event-subscriber-handle-next-step}
\end{code}

La méthode \Verb|handleStepError| (Code source~\ref{code:step-event-subscriber-handle-step-error}) est l'écouteur pour l'événement \Verb|StepErrorEvent|. Cet écouteur gère les erreurs survenues lors des maillons de traitement. Il récupère les informations d'erreur de l'événement et exécute les étapes de gestion des erreurs définies dans la configuration. Lorsqu'un événement \Verb|StepErrorEvent| est déclenché, la méthode \Verb|handleStepError| est appelée. Cet événement fournit un objet \Verb|Step| qui représente le maillon où l'erreur s'est produite. À partir de cet objet \Verb|Step|, la méthode identifie l'importation associée à l'erreur en récupérant son identifiant. Cela permet d'accéder à l'enregistrement de l'importation dans la base de données. En utilisant l'identifiant d'importation, la méthode récupère l'enregistrement de l'importation à partir de la base de données. Cela permet d'accéder aux détails de l'importation, tels que son statut actuel. La méthode met à jour le statut de l'importation pour indiquer qu'elle a été annulée (\Verb|UploadStatus::ABORTED|). Cela permet de notifier que l'importation a été interrompue en raison d'une erreur. Ensuite, la méthode vérifie si la configuration de l'importation contient une section dédiée aux actions en cas d'erreur (\Verb|ERROR_CONFIG_KEY|). Si une telle configuration est disponible, cela signifie qu'il existe des étapes spécifiques à exécuter en cas d'erreur. Si une configuration d'erreur est trouvée, la méthode récupère les classes des étapes d'erreur à partir de la configuration. Ces classes représentent les étapes spécifiques à effectuer pour gérer l'erreur. La méthode utilise la classe \Verb|Pipeline| de Laravel pour exécuter en séquence les étapes d'erreur. Elle envoie l'objet StepErrorEvent à travers la séquence de tâches d'erreur définies dans la configuration. Une fois que toutes les étapes d'erreur ont été exécutées, la méthode retourne true pour indiquer que le processus de gestion d'erreur est terminé avec succès.

\begin{code}
    \caption{La méthode \Verb{handleStepError} de la classe \Verb{StepEventSubscriber}.}
    \inputminted[firstline=53,lastline=72]{php}{code/StepEventSubscriber.php}
    \label{code:step-event-subscriber-handle-step-error}
\end{code}

La méthode \Verb|subscribe| (Code source~\ref{code:step-event-subscriber-subscribe}) attache les écouteurs aux événements spécifiques (\Verb|NextStepEvent| et \Verb|StepErrorEvent|). Cela définit quelle méthode doit être appelée lorsque ces événements sont déclenchés.

\begin{code}
    \caption{La méthode \Verb{subscribe} de la classe \Verb{StepEventSubscriber}.}
    \inputminted[firstline=74,lastline=85]{php}{code/StepEventSubscriber.php}
    \label{code:step-event-subscriber-subscribe}
\end{code}