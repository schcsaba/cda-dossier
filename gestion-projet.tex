\chapter{Gestion de projet}\label{ch:gestion-projet}

Pendant mon alternance, j'ai très vite compris que pour l'entreprise, il est important de maintenir son esprit innovant, de constamment générer davantage de valeur, et d'être à l'écoute tout en s'ajustant selon les exigences de sa clientèle. La politique d'innovation de l'entreprise repose sur les recommandations émanant à la fois de ses clients et de ses collaborateurs. Un comité dédié à l'innovation se rassemble hebdomadairement pour examiner les suggestions les plus récentes. Chaque concept est traité, trié et priorisé. Il s'agit d'un processus en plusieurs étapes qui implique également d'autres comités, comme l'illustre la Figure~\ref{fig:committees} et la Table~\ref{tblr:character-committees}. Par la suite, l'ensemble de ces propositions est transmis au département Recherche et Développement en vue de la création de nouvelles fonctionnalités. Tous les quatre mois, de nouvelles options viennent enrichir l'ensemble des services, désignées sous le terme \foreignquote{french}{Editions}.

\begin{figure}[ht]
    \centering
    \includegraphics[width=\textwidth]{img/committees}
    \caption{La participation des différents comités au processus de traitement des idées d'innovation.}
    \label{fig:committees}
\end{figure}

Une édition représente le fruit de quatre mois de travail de développement, cependant, son élaboration ne s'arrête pas là. Elle englobe la mise en place et la planification des activités de communication (marketing), la formation des équipes support et commerce, la rédaction de manuels et de tutoriels destinés aux clients, ainsi que la préparation des prochaines éditions à venir.

\begin{longtblr}[
    caption={Les caractéristiques des différents comités.},
    label={tblr:character-committees}
    ]{
    hlines,vlines,
    rowspec={Q[m,font=\footnotesize\bfseries,gray9]*{7}{Q[m,font=\footnotesize]}}
    }
    {Nom du        \\ comité}       & {Récur-\\rence}   & Objectif          & Entrées & Sorties \\
    {Comité        \\ d'innovation} & {Hebdo-\\madaire} & {Première analyse                     \\ des idées exprimées                     \\ puis étude de l'intérêt \\ de chaque idée.} & {Liste des \\ idées} & {Idées clarifiées, \\ note par niveau \\ d'intérêt} \\
    {Comité        \\ d'innovation \\ avancé} & Mensuel & {Clarifier les \\ spécifications \\ fonctionnelles des \\ idées exprimées au \\ cours des comités \\ d'innovations.} & {Ordre du jour \\ des idées, déjà \\ analysées} & {Réponses aux \\ questions \\ à l'ordre du \\ jour, ajoutées \\ aux backlog} \\
    {Comité        \\ fonctionnel \\ exceptionnel} & {Excep-\\tionnel} & {Le PO sollicite une \\ partie prenante \\ identifiée, qui devrait \\ lui permettre de \\ lever un certain \\ nombre de questions \\ autour d'une \\ fonctionnalité.} & {Fonctionnalités, \\ déjà analysées \\ et comportant \\ toujours \\ des questions} & {Les réponses \\ sont ajoutées \\ aux \\ fonctionnalités} \\
    {Comité        \\ tech} & {En fonction \\ des items \\ en attente \\ de chiffrage \\ dans le \\ backlog. \\ Au moins \\ hebdo-\\madaire.} & {Une présentation \\ des items, validés \\ fonctionnellement, \\ pour faire ressortir \\ un macro-chiffrage, \\ réalisé par l'équipe \\ d'expert.} & {Fonctionnalités \\ prêtes du \\ backlog \\ produit} & {Macro-chiffrage \\ ou questions \\ fonctionnelles} \\
    {Comité        \\ pilotage} & {Bimestriel} & {Donner une vision \\ claire de l'avancement \\ du travail réalisé \\ pour les services \\ concernés.} & {Indicateur clé \\ de performance \\ à partager} & {Adaptations \\ à mettre en \\ œuvre} \\
    {Comité        \\ d'édition} & {Bimestriel} & {Établir une \\ constitution \\ d'édition à partir \\ des idées prêtes. \\ En déduire un \\ objectif d'édition \\ permettant de \\ fédérer autour \\ d'une réalisation.} & {Fonctionnalités \\ prêtes, analyse du \\ temps disponible \\ faite à partir \\ des paramètres \\ vélocité / nombre \\ de demandes \\ courantes / dette \\ technique \dots
    } & {Liste des \\ composants \\ d'édition, \\ objectif \\ d'édition} \\
    {Comité        \\ présentation \\ technique /\\ Kickoff \\ Edition} & {En début \\ d'édition} & {Une présentation des \\ items embarqués dans \\ l'édition, ainsi qu'un \\ focus sur l'objectif \\ d'édition.} & {L'objectif, \\ le contenu \\ d'édition} & {Le retour \\ de l'équipe \\ sur le contenu \\ et l'objectif}
\end{longtblr}

\subsection{Méthodologie Agile}\label{subsec:agile}

L'équipe de développement de la géolocalisation de SuiviDeFlotte -- comme les autres équipes de développement de l'entreprise -- travaille selon la méthodologie agile SCRUM. Cette méthodologie est une approche de gestion de projet qui met l'accent sur la flexibilité, la collaboration et la livraison continue. Elle est largement utilisée dans le développement de logiciels et peut également être appliquée à d'autres domaines. SCRUM divise un projet en cycles appelés ``itérations'' ou ``sprints'' de courte durée, généralement de deux à quatre semaines, pendant lesquels une partie du travail est accomplie et livrée.

Les termes clés dans la méthodologie SCRUM comprennent :

\begin{description}
    \item[Product Owner (Propriétaire du Produit)] La personne responsable de définir et de prioriser les éléments du produit à développer. Le Propriétaire du Produit représente les besoins des utilisateurs et des parties prenantes.
    \item[Scrum Master (Maître de Scrum)] Le facilitateur du processus SCRUM. Le Scrum Master s'assure que l'équipe suit les principes SCRUM, élimine les obstacles et favorise un environnement de travail efficace.
    \item[Équipe de Développement] Le groupe de professionnels chargé de concevoir, développer, tester et livrer les éléments du produit à la fin de chaque sprint.
    \item[Product Backlog (Carnet de Produit)] Une liste dynamique et priorisée de toutes les fonctionnalités, tâches et améliorations potentielles du produit. Elle est gérée par le Propriétaire du Produit.
    \item[Sprint Planning (Planification de Sprint)] Une réunion au début de chaque sprint au cours de laquelle l'équipe choisit les éléments du Carnet de Produit à inclure dans le sprint et planifie comment les livrer.
    \item[Sprint Backlog (Carnet de Sprint)] La liste des éléments du Carnet de Produit choisis pour le sprint en cours, ainsi que les tâches nécessaires à leur réalisation.
    \item[Daily Scrum (Mêlée Quotidienne)] Une courte réunion quotidienne pendant laquelle l'équipe de développement partage les progrès réalisés depuis la dernière réunion, discute des tâches à accomplir et identifie les obstacles potentiels.
    \item[Sprint Review (Revue de Sprint)] Une réunion à la fin de chaque sprint au cours de laquelle l'équipe présente les éléments terminés et recueille les commentaires du Propriétaire du Produit et des parties prenantes.
    \item[Sprint Retrospective (Rétrospective de Sprint)] Une réunion après la Revue de Sprint au cours de laquelle l'équipe examine le déroulement du sprint et identifie les points forts et les points à améliorer.
    \item[User Story (Histoire Utilisateur)] Une Histoire Utilisateur est une courte description d'une fonctionnalité ou d'un aspect du produit, racontée du point de vue de l'utilisateur. Elle suit généralement le format ``En tant que [utilisateur], je veux [action] afin de [objectif]''. Les Histoires Utilisateurs sont des éléments du Carnet de Produit et aident à définir les fonctionnalités du produit du point de vue de l'utilisateur.
    \item[Story Point (Point d'Histoire)] Le Point d'Histoire est une unité relative utilisée pour estimer la complexité, l'effort et la taille des Histoires Utilisateurs ou des tâches de développement. Il n'a pas de valeur absolue, mais il sert à comparer la difficulté relative entre différentes Histoires Utilisateurs. Les équipes de développement attribuent des points d'histoire lors des estimations, ce qui les aide à planifier la quantité de travail qu'elles peuvent accomplir dans un sprint donné.
    \item[Epic (Épique)] Un Épic est une unité de travail plus large que les Histoires Utilisateurs individuelles. Il représente généralement un ensemble de fonctionnalités, de tâches ou de travaux qui sont trop importants pour être traités dans un seul sprint. Les Épics sont souvent des objectifs à long terme qui sont décomposés en Histoires Utilisateurs plus petites et gérables. Ils aident à organiser et à structurer le développement du produit en regroupant des éléments liés autour d'un thème ou d'un objectif commun. Les Épics sont inclus dans le Carnet de Produit et sont priorisés en fonction de leur valeur pour l'utilisateur et du contexte global du projet.
\end{description}

SCRUM encourage la transparence, l'adaptabilité et la collaboration continue entre les membres de l'équipe et les parties prenantes, ce qui permet de s'adapter aux changements et de fournir rapidement de la valeur tout au long du projet.

Dans l'équipe Géoloc, le processus suit un calendrier de sprints de deux semaines. Chaque cycle commence par un ensemble de réunions clés qui ont lieu tous les deuxièmes mardis. Cette journée englobe la Revue de Sprint, la Rétrospective de Sprint et la Planification de Sprint. Lors de la Revue de Sprint, les éléments achevés sont présentés au Propriétaire du Produit et aux parties prenantes, les retours sont recueillis et les priorités sont ajustées si nécessaire. La Rétrospective de Sprint offre l'opportunité à l'équipe de réfléchir aux succès et d'identifier les domaines à améliorer, favorisant une culture d'amélioration continue. Ensuite, la Planification de Sprint implique la sélection des Histoires Utilisateurs à inclure dans le sprint à venir, en tenant compte de leur complexité et de leur priorité.

Chaque jour ouvrable débute par une Mêlée Quotidienne de 15 minutes, au cours de laquelle les progrès depuis la dernière réunion sont passés en revue, les tâches sont discutées et les obstacles potentiels sont identifiés. Ces réunions se déroulent généralement en ligne via Google Meet pour accueillir les collègues travaillant à distance et garantir la participation de tous les membres de l'équipe, peu importe leur emplacement.

Actuellement, les rôles du Propriétaire du Produit et du Maître de Scrum ont été temporairement fusionnés en une seule personne. Cette configuration permet au Propriétaire du Produit de gérer les responsabilités du projet tout en facilitant le processus SCRUM, en maintenant une communication fluide avec l'équipe de développement.

Pour l'estimation de la complexité des tâches, des Points d'Histoire basés sur la séquence de Fibonacci (1, 2, 3, 5, 8, 13) sont utilisés. Cette approche aide à attribuer des valeurs relatives à différentes tâches et assure une évaluation cohérente de la charge de travail.

Enfin, pour maintenir la communication et la cohésion de l'équipe, les réunions en personne sont maintenues pour les mardis tous les deuxièmes semaines. Cette pratique encourage les échanges directs et la collaboration étroite entre les membres de l'équipe. Cette approche garantit l'alignement et facilite la résolution rapide de tout problème ou obstacle pouvant survenir.
\section[Jira : L'outil essentiel pour la gestion de projets]{Jira : L'outil essentiel pour la gestion quotidienne de projets}\label{sec:jira}

Au sein de l'entreprise, les équipes SCRUM se servent de Jira afin de consigner et de suivre tous les aspects de leur travail.

Jira est un outil essentiel pour une équipe SCRUM car il facilite la gestion complète du processus de développement agile. Il permet aux équipes de collaborer de manière efficace et de suivre chaque étape du cycle de vie du projet. Avec Jira, les équipes SCRUM peuvent créer, organiser et hiérarchiser leur Carnet de Produit en ajoutant des Histoires Utilisateurs, des tâches, des Épics et des bugs.

L'outil facilite la planification des sprints en permettant aux équipes de sélectionner les éléments du Carnet de Produit à inclure dans chaque itération. Les équipes peuvent estimer la complexité des tâches à l'aide de Story Points et suivre leur progression au fil du temps.

Les fonctionnalités de suivi des tâches et des problèmes dans Jira aident les équipes à gérer leur travail quotidien. Chaque membre de l'équipe peut mettre à jour l'état de ses tâches, signaler les obstacles et collaborer de manière transparente avec les autres membres de l'équipe.

Les réunions SCRUM telles que la Mêlée Quotidienne, la Revue de Sprint et la Rétrospective de Sprint peuvent être orchestrées efficacement grâce à Jira. L'outil permet de suivre les progrès, de partager les résultats et de documenter les réflexions pour chaque itération.

Les flux de travail (workflows) dans Jira constituent un mécanisme central pour orchestrer et suivre le déroulement des tâches et des projets. Ces flux définissent les étapes spécifiques à suivre pour qu'une tâche ou un problème progresse, de sa création à son achèvement. Les équipes peuvent personnaliser ces flux pour refléter leurs processus uniques, en définissant les transitions entre les étapes et en assignant des responsabilités à différents membres de l'équipe. Les flux de travail de Jira contribuent à maintenir la transparence, à améliorer l'efficacité et à garantir que toutes les parties prenantes restent informées de l'avancement du travail. Grâce à cette fonctionnalité, les équipes peuvent gérer avec agilité et précision les tâches, tout en favorisant une collaboration fluide et une visibilité accrue sur les projets.

En somme, Jira est un atout précieux pour une équipe SCRUM car il simplifie et rationalise le processus de développement agile en offrant une plateforme centralisée pour la planification, la collaboration, le suivi et l'amélioration continue.
\section{Versioning et les environnements}\label{sec:versioning-environments}

Dans l'entreprise, les équipes de développement utilisent Git pour la gestion des versions et GitLab pour gérer le processus de développement.

Git est un système de gestion de version décentralisé largement utilisé dans le développement de logiciels. Il permet aux équipes de collaborer efficacement sur des projets en suivant les modifications apportées aux fichiers au fil du temps. Grâce à Git, les développeurs peuvent créer des branches pour travailler sur des fonctionnalités spécifiques ou des corrections de bugs sans perturber le code principal. Les commits, qui représentent des enregistrements de changements, sont la pierre angulaire de Git, permettant de garder une trace claire de l'évolution du code.

GitLab, quant à lui, est une plateforme de gestion de développement logiciel basée sur Git. Elle offre un environnement complet pour le cycle de vie du développement, de la planification à la surveillance. GitLab permet aux équipes de suivre les problèmes, de planifier les sprints, de gérer les demandes d'extraction et de créer des pipelines d'intégration continue pour automatiser les tests et le déploiement. En regroupant toutes ces fonctionnalités au même endroit, GitLab facilite la collaboration entre les membres de l'équipe et permet une gestion transparente et efficace des projets de développement.

Parmi ces nombreuses fonctionnalités, les équipes de DevOps et de développement n'utilisent cependant pas les fonctionnalités de gestion de projet et d'intégration continue/de livraison continue de GitLab, puisqu'elles utilisent Jira (comme nous l'avons vu dans la section précédente) et TeamCity (comme nous le verrons dans la section suivante) à ces fins. GitLab est donc principalement utilisé comme un hébergeur de dépôt de code et un outil de révision de code avec des fonctionnalités telles que les demandes de fusion (merge request).

La Figure~\ref{fig:versioning-and-environments} illustre les principes du versioning et les différents environnements utilisés pour héberger les différentes versions du code. La Table~\ref{tblr:environments} apporte quelques compléments d'information.

Du point de vue des versions et des environnements, le processus de développement entre deux éditions est le suivant. Avant la sortie publique de la nouvelle édition, il y a toujours une sortie interne. La sortie interne est créée par la fusion de la branche \mintinline{console}{develop} dans la branche \mintinline{console}{master} avec la balise \mintinline{console}{{numéro d'édition}.0}. Après cela, une nouvelle branche de \mintinline{console}{release} est créée à partir de la branche \mintinline{console}{master} avec le nom \mintinline{console}{release/{numéro d'édition}}. Cette nouvelle branche \mintinline{console}{release} n'est pas touchée avant la sortie publique.

Entre la sortie interne et la sortie publique, des correctifs urgents (\mintinline{console}{hotfix}) peuvent être apportés. Lorsque le développeur travaille sur un \mintinline{console}{hotfix}, il crée d'abord une branche à partir de l'ancienne branche \mintinline{console}{release} avec un nom \mintinline{console}{hotfix/{bogue}}, il y dépose ses modifications en travaillant dans son environnement sandbox développeur, puis il crée une demande de fusion pour réintégrer cette branche dans l'ancienne branche \mintinline{console}{release}. Lorsqu'une décision positive est prise concernant la sortie publique, l'ancienne branche \mintinline{console}{release} est fusionnée dans les branches \mintinline{console}{master} et \mintinline{console}{develop}, l'ancienne branche \mintinline{console}{release} est clôturée (archivée), la branche \mintinline{console}{master} est fusionnée dans la nouvelle branche \mintinline{console}{release}, puis la nouvelle branche \mintinline{console}{release} est fusionnée dans la branche \mintinline{console}{develop}, enfin la branche \mintinline{console}{master} est mise en production dans l'environnement de production avec la balise \mintinline{console}{{numéro d'édition}.1}. Pendant la période entre la sortie interne et la sortie publique, les développeurs peuvent travailler sur de nouvelles fonctionnalités, mais celles-ci ne seront pas intégrées dans l'édition actuelle, elles ne pourront l'être que dans la prochaine.

Après la sortie publique de la nouvelle édition, les développeurs commencent à travailler sur la prochaine édition. Lorsqu'ils travaillent sur des bogues, des correctifs (\mintinline{console}{fix}), ils créent une branche à partir de la branche \mintinline{console}{release} avec un nom \mintinline{console}{fix/{bogue}}, ils travaillent dans leur environnement sandbox, et lorsqu'ils ont terminé, ils créent une demande de fusion pour fusionner leur branche dans la branche \mintinline{console}{release}. Lorsqu'ils travaillent sur des fonctionnalités, cela se passe de la même manière, sauf qu'ils créent leur branche à partir de la branche \mintinline{console}{develop} avec un nom \mintinline{console}{feat/{fonctionnalité}} et bien sûr ils créent la demande de fusion pour fusionner leur branche dans la branche \mintinline{console}{develop}. Au cours de cette période, les corrections peuvent être fusionnées dans les branches \mintinline{console}{master} et \mintinline{console}{develop} et elles peuvent être mises en production si nécessaire à la fin des sprints ou même au milieu de ceux-ci. Dans ce cas, une nouvelle subversion est ajoutée au commit de fusion sous la forme d'une balise (\mintinline{console}{{numéro d'édition}.{numéro d'increment}}).

\begin{sidewaysfigure}
    \centering
    \includegraphics[width=\textwidth]{img/versioning-and-environments}
    \caption{Le schéma du versioning et les environnements.}
    \label{fig:versioning-and-environments}
\end{sidewaysfigure}

\begin{longtblr}[
    caption={Les caractéristiques des différents environnements.},
    label={tblr:environments}
    ]{
    hlines,vlines,
    rowspec={Q[m,font=\footnotesize\bfseries,gray9]*{5}{Q[m,font=\footnotesize]}}
    }
    Environnement    & {Pour quels                       \\ services} & Branches & {Type de \\ comits} & {Déploiement \\ TeamCity} \\
    {sandbox                                             \\ développeur} & Développeur & {feat/\\\{fonction-\\nalité\} \\ fix/\{bogue\} \\ hotfix/\{bogue\}} &  & -- \\
    recette-develop  & {Le PO vérifie                    \\ les fonction-\\nalités}                                & develop             & feat                                                         & Auto                                                                      \\
    recette-releases & {Le PO vérifie                    \\ les corrections}                                    & releases/édition    & fix, hotfix                                                          & Auto                                                                      \\
    recette-master   & {Autres                           \\ services                    \\ pour test \\ (direction, \\ marketing, \\ commerce \\ \dots)} & master              & {(env juste \\ avant la prod, \\ préproduction)}                     & {Manuel, \\ section master \\ dans TeamCity, \\ demande via \\ ticket MEP \\ à DevOps}     \\
    production       &                & tag & {Commit en \\ production \\ avec tag \\ \{numéro d'édtion\}.\\\{numéro \\ d'increment\}} & {Manuel, \\ section \\ production \\ dans TeamCity, \\ demande via \\ ticket MEP \\ à DevOps}
\end{longtblr}
\section{Intégration Continue et Déploiement Continu}\label{sec:ci-cd}

L'équipe Infrastructure/DevOps utilise TeamCity qui est une plateforme d'intégration continue et de livraison continue largement adoptée dans le domaine du développement logiciel. Elle permet aux équipes de développeurs de automatiser le processus de construction, de test et de déploiement de leurs applications. TeamCity offre un environnement convivial où les développeurs peuvent configurer des pipelines d'intégration continue en spécifiant les étapes nécessaires, telles que la compilation du code, les tests unitaires et les déploiements sur différents environnements.

Une caractéristique clé de TeamCity est sa capacité à détecter automatiquement les changements dans le code source et à déclencher des builds et des tests en conséquence. Cela permet aux équipes de détecter rapidement les problèmes et de s'assurer que le code reste stable et fonctionnel à chaque étape du développement.

En outre, TeamCity offre des fonctionnalités avancées telles que la gestion des agents de construction, la parallélisation des tâches, la gestion des paramètres de build et des rapports détaillés sur les résultats des tests. Cette plateforme aide les équipes à maintenir un processus de développement fluide et à garantir la qualité du code grâce à l'automatisation et à la surveillance continue.

Comme l'indique la Table~\ref{tblr:environments} (page~\pageref{tblr:environments}), le déploiement des branches \mintinline{console}{develop} et \mintinline{console}{release} dans les environnements recette-develop et recette-releases est automatique. Lors d'un commit ou d'un merge dans la branche \mintinline{console}{develop} ou \mintinline{console}{release}, TeamCity intercepte l'événement et lance automatiquement les actions ci-dessous :

\begin{itemize}
    \item Compilation du code
    \item Exécution des tests unitaires
    \item Création des fichiers de publication
    \item Création du package de publication
    \item Upload de la mise à jour sur Lambda
    \item Création d'une version et déploiement sur l'environnement correspondant
\end{itemize}

Ces actions s'enchaînent sauf en cas d'erreur dans l'une d'elle. Dans ce cas, la build est en échec et les actions s'interrompent.

Pour la branche \mintinline{console}{master} et l'environnement de production, le processus de déploiement est lancé manuellement par l'équipe DevOps.

Pendant mon alternance, je n'ai pas eu accès au portail TeamCity, parce qu'il était géré et utilisé par l'équipe DevOps. En plus, je travaillais principalement sur un tout nouveau projet dont le déploiement n'était pas encore intégré dans TeamCity. Je ne l'ai donc pas vu de l'intérieur. Cependant, j'ai vu les résultats de son fonctionnement lorsque j'ai travaillé parfois sur d'autres projets (ex. Portail, API, Gestion de Parc, Cores) qui ont été déployés par TeamCity. Lorsque je travaillais sur des bogues ou des fonctionnalités et que ma branche était fusionnée dans la branche \mintinline{console}{release} ou \mintinline{console}{develop}, je voyais le message dans Slack (qui est la plateforme de communication utilisée par les équipes R\&D) que le déploiement avait commencé et plus tard un autre message sur le succès ou l'échec du déploiement. Après un déploiement réussi, bien sûr, j'ai vu aussi que les modifications que j'ai faites ont été appliquées dans l'environnement recette-releases ou recette-develop.

\subsection{Le déploiement des paquets de projet Cores}

Le déploiement des paquets PHP dans le projet Cores est bien sûr différent du déploiement des autres projets. La procédure est la suivante :

\begin{enumerate}
    \item \textbf{Le développeur met à jour le paquet dans GitLab} : Un développeur apporte des modifications et des mises à jour au code du paquet, éventuellement en corrigeant des bogues ou en ajoutant de nouvelles fonctionnalités. Il enregistre ces modifications dans le dépôt GitLab.
    \item \textbf{Augmentation de la version du paquet Composer} : Dans le fichier \mintinline{console}{composer.json} du paquet, le développeur incrémente le numéro de version du paquet. Il s'agit d'une étape importante pour que Composer reconnaisse qu'une mise à jour a eu lieu.
    \item \textbf{Notifier Satis} : Déclencher le processus de mise à jour pour Satis en exécutant le script créé à cet effet.
    \item \textbf{Processus de mise à jour de Satis} : Le déclenchement du processus de mise à jour de Satis implique l'exécution d'une commande qui régénère les métadonnées statiques de Composer pour le dépôt Satis. Cette commande récupère les dernières informations du dépôt GitLab et met à jour les métadonnées en conséquence.
    \item \textbf{Paquet mis à jour dans le dépôt Satis} : Le paquet mis à jour avec sa nouvelle version est maintenant inclus dans le dépôt Satis. Cela signifie que le paquet mis à jour est maintenant disponible pour l'installation via Composer à partir du dépôt privé Satis.
    \item \textbf{Composer Update} : Les développeurs travaillant sur des projets qui dépendent de ce paquet mis à jour peuvent maintenant utiliser Composer pour mettre à jour leurs dépendances. Lorsqu'ils lancent composer update ou composer require, Composer vérifie le dépôt Satis, trouve la nouvelle version du paquetage mis à jour, le télécharge et l'installe dans leur projet.
\end{enumerate}

Comme nous pouvons le constater, ce processus n'est pas encore automatisé à l'exception du fichier script pour notifier Satis. Dans ce processus manuel, il est de la responsabilité du développeur ou de l'équipe DevOps de notifier le Satis et de déclencher la mise à jour.

\subsection{Le déploiement du projet Pipeline documentaire}

Comme je l'ai mentionné précédemment, étant un nouveau projet, le déploiement du projet Pipeline documentaire n'est pas encore intégré dans TeamCity, il est fait de manière semi-automatique par l'équipe DevOps. Il peut être qualifié de semi-automatique car le projet fonctionne dans des conteneurs Docker\footnote{Docker est une plateforme de virtualisation légère et portable qui permet de créer, déployer et exécuter des applications dans des conteneurs. Les conteneurs sont des environnements isolés qui regroupent tous les éléments nécessaires à l'exécution d'une application, tels que le code, les bibliothèques et les dépendances. Cela garantit une cohérence entre les environnements de développement, de test et de production, simplifiant ainsi le processus de déploiement.} pendant le développement ainsi qu'en production, il a donc un \mintinline{console}{Dockerfile},\footnote{Un \mintinline{console}{Dockerfile} est un fichier texte qui contient les instructions pour construire une image Docker. Une image Docker est un modèle d'environnement qui contient tous les composants nécessaires pour exécuter une application. Le \mintinline{console}{Dockerfile} décrit les étapes pour installer les dépendances, copier les fichiers de l'application et configurer l'environnement. Une fois le \mintinline{console}{Dockerfile} créé, il peut être utilisé pour générer une image Docker prête à être exécutée dans un conteneur.} un fichier \mintinline{console}{docker-compose.yaml}\footnote{Le fichier \mintinline{console}{docker-compose.yaml} est un autre élément clé dans l'écosystème Docker. Il permet de définir et de gérer plusieurs conteneurs interconnectés en tant qu'application unique. Ce fichier décrit les services, les images, les ports exposés et les liens entre les conteneurs, facilitant ainsi le déploiement et la gestion d'applications complexes à plusieurs composants.} et l'équipe DevOps a créé un fichier de script d'installation aussi.