\subsection{Migrations de la base de données}

Les migrations dans Laravel sont un mécanisme permettant de gérer et de maintenir la structure de la base de données d'une application de manière efficace et contrôlée. Les migrations définissent les schémas des tables de la base de données, ainsi que les modifications ultérieures à ces schémas. Elles offrent un moyen de collaborer entre développeurs en suivant un processus de versionnement pour les modifications de la base de données.

Pour créer une migration, on utilise l'outil de ligne de commande Artisan fourni par Laravel, en exécutant la commande \Verb|php artisan make:migration|. Cette commande génère un fichier de migration dans le répertoire \Verb|database/migrations| du projet. Dans ce fichier, on peut définir les colonnes de la table, les clés étrangères, les index, etc.

Une fois la migration créée, on peut exécuter la commande \Verb|php artisan migrate| pour appliquer les migrations en attente et mettre à jour la base de données selon les schémas définis. Les migrations permettent également de revenir en arrière en cas de besoin avec la commande \Verb|php artisan migrate:rollback|.

L'utilisation des migrations permet aux développeurs de travailler de manière collaborative et organisée en maintenant un historique des changements de structure de la base de données. Cela facilite également le déploiement sur différents environnements et assure une gestion centralisée de la structure de la base de données au sein du code source du projet.

Comme le modèle physique de données (Figure~\ref{fig:pmd}, page~\pageref{fig:pmd}) l'a défini, nous avons créé trois tables dans la base de données à l'aide de fichiers de migration : la table des téléchargements (\Verb|uploads|), la table des maillons des téléchargements (\Verb|upload_step|) et la table des résultats des téléchargements \Verb|upload_results|. À titre d'exemple, le fichier de migration de la table \Verb|uploads| est présenté dans le Code source~\ref{code:uploads-migration}.

Cette migration crée une table nommée \Verb|uploads| avec plusieurs colonnes :

\begin{itemize}
    \item \textbf{\Verb|id|} : Une clé primaire unique générée à l'aide d'ULID (Universally Unique Lexicographically Sortable Identifier).
    \item \textbf{\Verb|parameters|} : Une colonne de type JSON pour stocker des paramètres au format JSON.
    \item \textbf{\Verb|client_app|} : Une colonne de type entier pour enregistrer l'application cliente.
    \item \textbf{\Verb|upload_type|} : Une colonne de type chaîne de caractères (string) pour spécifier le type de téléchargement.
    \item \textbf{\Verb|status|} : Une colonne de type entier avec une valeur par défaut de 0, représentant l'état du téléchargement (0 : reçu, 1 : démarré, 2 : en attente, 3 : terminé, 4 : abandonné).
    \item \textbf{\Verb|timestamps|} : Deux colonnes pour enregistrer les horodatages de création et de mise à jour des enregistrements.
    \item \textbf{\Verb|deleted_at|} : Une colonne pour prendre en charge la suppression douce (soft delete) avec une précision temporelle de 0.
\end{itemize}

\begin{code}
    \caption{La classe de migration de la table \Verb{uploads}.}
    \inputminted{php}{code/2022_12_20_111202_create_uploads_table.php}
    \label{code:uploads-migration}
\end{code}

La méthode \Verb|up| est utilisée pour créer la table avec les colonnes spécifiées, tandis que la méthode \Verb|down| est utilisée pour supprimer la table en cas de rollback. Le code est encapsulé dans une classe anonyme héritant de la classe de migration. Cela permet de créer et exécuter la migration sans avoir à nommer explicitement la classe de migration, ce qui est courant dans les fichiers de migration Laravel.

L'objet \Verb|table|, au sein de la classe de migration, représente l'entité qui permet de définir et de structurer la table de base de données au moyen de la migration. Dans le contexte de Laravel, cet objet est une instance de la classe \Verb|Blueprint|, qui offre un ensemble de méthodes et de fonctionnalités pour concevoir la structure de la table de manière programmatique.

Chaque méthode appelée sur l'objet \Verb|table| correspond à une opération spécifique permettant de construire la table. Ceci facilite la création cohérente et reproductible de la base de données. Par exemple, dans ce cas, les instructions telles que \Verb|$table->ulid('id')->primary()| définissent la colonne \Verb|id| comme une clé primaire dotée d'une valeur unique générée par le biais d'ULID. De la même manière, d'autres méthodes comme \Verb|$table->json('parameters')|, \Verb|$table->integer('client_app')|, etc., définissent les colonnes de la table et leurs types de données respectifs.

    L'objet \Verb|table| revêt une importance fondamentale au sein du processus de migration Laravel. Il agit comme un canal permettant de décrire, de manière systématique, la structure de la table de base de données. Ce mécanisme offre une approche structurée et organisée pour gérer les changements de schéma de la base de données de manière réversible et documentée.

    La structure des fichiers de migration des deux autres tables est similaire à celle-ci, avec des colonnes différentes bien sûr. Ces deux tables sont dans une relation un-à-plusieurs avec la table \Verb|uploads|, donc elles ont des clés étrangères comme l'une de leurs colonnes créées par le code suivant dans leurs fichiers de migration : \Verb|$table->foreignUlid('upload_id')->constrained('uploads');|.

Laravel utilise les informations dans des fichiers des migrations pour générer automatiquement les requêtes SQL nécessaires. Cela se traduit par la création, la modification ou la suppression des tables et des colonnes dans la base de données sous-jacente. Par exemple, le code SQL généré par Laravel à partir du fichier de migration de la table \Verb{uploads} est présenté dans le Code source~\ref{code:uploads-sql}.

\begin{code}
    \caption{Le code SQL généré par Laravel sur la base du fichier de migration de la table \Verb{uploads}.}
    \inputminted{sql}{code/uploads.sql}
    \label{code:uploads-sql}
\end{code}