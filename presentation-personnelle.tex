\section{Présentation personnelle}\label{sec:presentation-personnelle}

\selectlanguage{french}

Je m'appelle Csaba Schnitchen, j'ai 46 ans. Je suis né en Hongrie, où j'ai obtenu un diplôme en biologie/écologie en 2002, suivi d'un doctorat en sciences de l'environnement en 2007. Jusqu'en 2020, j'ai travaillé en tant que professeur et chercheur dans deux universités. Au cours de mes recherches en écologie, j'ai découvert la programmation et j'ai utilisé les langages R et Python pour créer différents modèles écologiques et des graphiques pour des publications scientifiques. Par la suite, j'ai appliqué mes connaissances en Python pour créer un site web universitaire en utilisant le framework Django.

En 2020, j'ai déménagé définitivement en France. Ici, j'ai pris la décision de changer de profession et de me lancer dans le domaine de l'informatique. En 2022, j'ai suivi la formation Développeur web et web mobile au CEFIM, ce qui a considérablement approfondi mes connaissances en informatique et en programmation, acquises en autodidacte. Par la suite, j'ai poursuivi mes études en suivant la formation Concepteur développeur d'applications, toujours au CEFIM. Cette formation se déroulait en alternance. Pour cela, j'ai eu la chance de rejoindre Setipp, une société française, basée à Tours, proposant des solutions de télécommunications, de sécurité des travailleurs isolés et de géolocalisation professionnelle/gestion de flotte, sur un rythme de trois semaines en entreprise et une semaine CEFIM.