\chapter{Project Summary}\label{ch:project-summary}

\selectlanguage{magyar}

SCHNITCHEN Csabának hívnak, 46 éves vagyok. Magyarországon születtem, ahol 2002-ben biológus/ökológus diplomát, majd 2007-ben doktori címet szereztem a környezettudományok területén. Ezt követően 2020-ig oktatóként és kutatóként dolgoztam két egyetemen. Az ökológusi kutatómunkám során találkoztam először a programozással, amikor is az R és a Python nyelv használatával különböző ökológiai modelleket, továbbá a tudományos publikációkba szánt ábrákat hoztunk létre. A Python nyelv terén szerzett tudásomat később arra is fel tudtam használni, hogy létrehozzak egy egyetemi weboldalt a Django keretrendszerben.

Franciaországba 2020-ban költöztem tartósan. Itt arra a döntésre jutottam, hogy szakmát váltok és megpróbálok az informatika területén elhelyezkedni. 2022-ben elvégeztem a CEFIM-ben a web- és mobilfejlesztői képzést, amely az addigi autodidakta módon szerzett informatikai és programozói ismereteimet jelentős mértékben elmélyítette. A tanulmányaimat ezután az alkalmazástervező és -fejlesztő képzésen folytattam tovább szintén a CEFIM-ben. Ez a képzés váltakozó rendszerben folyt oly módon, hogy az egy éves képzés során az egy hetes iskolai képzéseket három hétig tartó munkahelyi szakmai gyakorlatok váltották. Szakmai gyakorlatomat a Setipp nevű cégnél végeztem Toursban.

A Setipp három szolgáltatás- és termékcsaládot nyújt az ügyfeleinek három márkanév alatt: a Setipp, a Beepiz és a SuiviDeFlotte. A Setipp telekommunikációs szolgáltatásokat foglal magába cégek számára. A Beepiz webes és mobil alkalmazásokat nyújt, amelyek lehetővé teszik az elszigetelten dolgozó munkavállalók számára, hogy minden körülmények között biztonságban legyenek és vészhelyzeti szolgáltatásokat riaszthassanak. A SuiviDeFlotte professzionális geolokalizációs és flottakezelési megoldásokat kínál. Online szolgáltatásaik számos funkciója között szerepel, hogy segítenek a sofőröknek üzemanyagot spórolni, illetve segítenek a vállalatoknak eldönteni, hogy melyik autót érné meg elektromosra cserélni. Én e harmadik termékcsoporton belül dolgoztam a webfejlesztői csapatban.

A webfejlesztői csapaton belül a tutorom és én -- valamint a későbbiekben még egy hozzánk csatlakozó kolléga -- egy teljesen új modul, egy API kifejlesztését kaptuk feladatul. A dokumentum csővezeték nevű API célja az volt, hogy a cég más szolgáltatásai felől érkező különböző dokumentumokat, például üzemanyagvásárlások tranzakcióit tartalmazó szöveges fájlokat vagy beszkennelt számlákat fel tudja dolgozni, a bennük tárolt adatokat ki tudja nyerni, és el tudja tárolni azokat az adatbázisban. A modult úgy kellett felépítenünk, hogy a későbbiekben könnyen bővíthető legyen újabb és újabb fájltípusok feldolgozási képességével.

A modult a Laravel keretrendszerben készítettük el, az adatokat MariaDB adatbázisban tároltuk. A projekt során a flottakezelési webhelyen belül az üzemanyagvásárlási tranzakciókat tartalmazó fájlok feltöltésére szolgáló webes felhasználói felület átalakításán is dolgoztam adaptálva annak működését az új API-hoz. Ez a felület a Blade template-nyelvet és a Vue.js keretrendszert alkalmazta.

A webfejlesztői csapatban az agilis elvek szerint dolgoztunk, azon belül is a SCRUM módszert alkalmaztuk a munkánk megszervezéséhez, mely által mindig rugalmasan tudtunk alkalmazkodni a felmerülő igényekhez.