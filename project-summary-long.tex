\chapter{Project Summary}\label{ch:project-summary}

\selectlanguage{english}

My name is Csaba SCHNITCHEN, and I'm 46 years old. I was born in Hungary, where I obtained a degree in biology/ecology in 2002, followed by a doctoral degree in environmental sciences in 2007. Until 2020, I worked as a lecturer and researcher at two universities. During my ecological research work, I first encountered programming, using the R and Python languages to create various ecological models and figures for scientific publications. Later on, I applied my knowledge of Python to create a university website using the Django framework.

In 2020, I permanently moved to France. Here, I made the decision to change professions and try to establish myself in the field of computer science. In 2022, I completed the web and mobile development training at CEFIM, which significantly deepened my self-taught knowledge of IT and programming. Following this, I continued my studies in application design and development, also at CEFIM. This training was conducted in an alternating system, where one-week classroom training sessions were followed by three-week work-based internships. I completed my internship at a company called Setipp in Tours.

Setipp offers three service and product families to its clients under three brand names: Setipp, Beepiz, and SuiviDeFlotte. Setipp provides telecommunications services for companies. Beepiz offers web and mobile applications that allow remote workers to stay safe under all circumstances and access emergency services. SuiviDeFlotte offers professional geolocation and fleet management solutions. Their online services include functions such as helping drivers save fuel and assisting companies in deciding which vehicles are worth switching to electric. Within this third product group, I worked in the web development team.

Within the web development team, my tutor and I, along with a colleague who later joined us, were tasked with developing a completely new module, an API. The purpose of the API, called "Document Pipeline," was to process various documents coming from other services of the company, such as text files containing fuel purchase transactions or scanned invoices. It needed to extract and store the data from these documents in the database. The module had to be designed in a way that it could easily be expanded to handle processing of new file types in the future.

We developed the module using the Laravel framework and stored the data in a MariaDB database. During the project, I also worked on transforming the web user interface for uploading files containing fuel purchase transactions within the fleet management website to adapt its functionality to the new API. This interface utilized the Blade templating language and the Vue.js framework.

Within the web development team, we followed agile principles, specifically applying the SCRUM methodology to organize our work, which allowed us to adapt flexibly to emerging needs.