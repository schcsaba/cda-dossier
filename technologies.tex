\chapter{Technologies}\label{ch:technologies}

Au cours du projet Pipeline Documentaire, nous avons choisi diverses technologies pour répondre aux besoins de notre initiative (Figure~\ref{fig:technologies}). En suivant les principes agiles de la méthodologie Scrum, nous avons utilisé Jira pour documenter nos avancées et faciliter la collaboration. Nous avons rassemblé les informations essentielles liées au projet dans Con\-flu\-ence, ce qui a favorisé une meilleure compréhension et une communication fluide au sein de l'équipe.

Pour la gestion des versions, nous avons opté pour git et GitLab, permettant ainsi un suivi précis des modifications et une collaboration simplifiée entre les membres de l'équipe. Dans le domaine du développement, nous avons travaillé avec PHP\-Storm, qui nous a fourni un ensemble d'outils adaptés à nos besoins pour la création d'applications PHP.

Afin de faciliter le déploiement, nous avons utilisé Docker et Docker Compose pour la conteneurisation du projet, ce qui a simplifié la configuration et la maintenance de l'environnement. Le cœur de notre application a été construit avec PHP 8.1 et Laravel 9, offrant une base solide pour nos développements. Nous avons utilisé MariaDB 10 comme base de données en raison de sa stabilité et de ses performances. Pour le serveur web, nous avons choisi Nginx, qui a assuré la distribution efficace de notre application et des temps de réponse rapides. Pour tester notre API, nous avons fait appel à Postman, ce qui nous a permis d'effectuer des tests approfondis pour garantir le bon fonctionnement de chaque composant.

Dans le cadre du projet Gestion de parc (Figure~\ref{fig:architecture}), j'ai travaillé sur la page d'importation des transactions de carburant. Mon objectif était de mettre à jour l'API utilisée pour cette page en utilisant la nouvelle API de Pipeline Documentaire. Pour cela, J'ai réalisé le maquettage avec Figma et j'ai travaillé avec Vue.js et Blade dans le cadre de la mission frontend, en cherchant à offrir une expérience utilisateur améliorée et conviviale.

\begin{figure}[ht]
    \centering
    \includegraphics[width=0.9\textwidth]{img/technologies01}
    \caption{Les principales technologies utilisées dans le cadre du projet.}
    \label{fig:technologies}
\end{figure}