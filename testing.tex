\subsection{Testing}

Les tests unitaires et fonctionnels jouent un rôle crucial dans le processus de développement logiciel en garantissant la qualité, la stabilité et la fiabilité d'une application. Chacun de ces types de tests a une signification spécifique dans la manière dont ils contribuent à améliorer le code et à assurer son bon fonctionnement.

Les tests unitaires ont pour objectif de vérifier le bon fonctionnement d'unités individuelles de code, telles que des fonctions, des méthodes ou des classes. En isolant une unité de code et en testant séparément son comportement attendu, les tests unitaires permettent de détecter rapidement les erreurs et les bugs au niveau granulaire. Ils favorisent également la modularité en rendant chaque composant du code indépendant et testable. Les tests unitaires sont particulièrement utiles pour repérer les régressions potentielles lors des modifications de code, garantissant ainsi que les changements n'affectent pas les fonctionnalités existantes.

Les tests fonctionnels, d'autre part, sont axés sur l'évaluation du comportement global de l'application. Ils simulent des interactions réelles entre les différentes parties du système, en vérifiant si l'application fonctionne comme prévu pour l'utilisateur final. Les tests fonctionnels sont essentiels pour s'assurer que toutes les fonctionnalités, interfaces et flux de travail sont corrects et cohérents. Ils détectent les problèmes qui pourraient survenir lorsque plusieurs composants interagissent, offrant ainsi une vision plus complète de la qualité de l'application.

En combinant ces deux approches de tests, les développeurs peuvent bénéficier d'une couverture de test solide qui identifie et résout les problèmes à différents niveaux. Les tests unitaires offrent une vérification détaillée au niveau du code source, tandis que les tests fonctionnels fournissent une assurance plus large quant à l'expérience utilisateur et à la conformité globale de l'application. En travaillant conjointement, ces deux types de tests renforcent la confiance dans la stabilité et la performance de l'application, ce qui contribue à réduire les erreurs, à améliorer la maintenance du code et à offrir une expérience utilisateur optimale.

Dans Laravel, la création de tests unitaires et fonctionnels est grandement facilitée grâce à l'utilisation de la bibliothèque de tests intégrée. Pour créer des tests dans Laravel, on peut exécuter la commande suivante : \Verb|php artisan make:test NomDeTest|. Cela créera un fichier de test dans le dossier \Verb|tests/Feature|. Si on veut créer un test dans le dossier \Verb|tests/Unit|, il faut utiliser l'option \Verb|--unit| lors de l'exécution de la commande. Dans la suite, je montrerai un exemple de test unitaire et un exemple de test fonctionnel que nous avons créé.

\subsubsection{Tests unitaires}

Pour les tests unitaires des maillons, nous avons créé une classe abstraite appelée StepTest (Code source~\ref{code:step-test}) qui peut être sous-classée pour construire les tests unitaires spécifiques.

\begin{code}
    \caption{Le classe \Verb{StepTest}.}
    \inputminted{php}{code/StepTest.php}
    \label{code:step-test}
\end{code}

La classe utilise le trait \Verb|RefreshDatabase| pour s'assurer que la base de données est réinitialisée entre les tests. Elle a une propriété privée \Verb|upload| qui représente un modèle de données simulé pour un \Verb|Upload|. Le nom du fichier d'entrée est également stocké.

La méthode \Verb|setUp| est exécutée avant chaque test et crée un objet \Verb|Upload| simulé, tandis que la méthode \Verb|tearDown| est exécutée après chaque test pour réinitialiser les simulations de stockage.

La classe fournit plusieurs méthodes pour effectuer des actions spécifiques lors des tests, telles que la création d'un fichier d'entrée, la récupération du résultat du maillon et la récupération du contenu du fichier journal.

En utilisant cette classe abstraite, des tests spécifiques à chaque maillon du processus de traitement peuvent être créés. Ces tests peuvent simuler l'entrée, exécuter du maillon, et ensuite vérifier les résultats attendus, comme le contenu des fichiers résultants et les journaux.

En résumé, cette classe abstraite de tests unitaires facilite la création de tests pour les différents maillons de traitement d'importation de fichiers dans l'application. Cela permet de s'assurer que chaque maillon du processus fonctionne correctement et produit les résultats attendus.

La classe \Verb|CheckEmailTest| représente un test unitaire pour le maillon \Verb|CheckEmail|. La méthode \Verb|setUp| est utilisée pour configurer le contexte de test. Elle hérite de la méthode parent \Verb|setUp|, qui prépare la base de données de test et crée un fichier factice de travail. Dans la méthode \Verb|setUp|, des données factices contenant des adresses e-mail incorrectes sont créées et enregistrées dans le fichier d'entrée. La méthode de test \Verb|testCheckEmailWithWrongEmail| est définie pour tester le comportement de la classe \Verb|CheckEmail| lorsqu'elle rencontre une adresse e-mail incorrecte dans le fichier d'entrée. Dans cette méthode de test, des paramètres de configuration sont définis, tels que le séparateur de champ (\Verb|,|) et le champ e-mail (\Verb|e-mail*|). Ensuite, une instance de la classe \Verb|CheckEmail| est créée avec les paramètres appropriés, y compris l'ID de téléchargement, le type de téléchargement, les paramètres de configuration, et le nom du fichier d'entrée. La méthode \Verb|assertThrows| est utilisée pour vérifier si l'exécution du processus \Verb|process| de la classe \Verb|CheckEmail| génère une exception de type Exception. Si une exception est effectivement générée, cela signifie que le test réussit, car il s'attend à ce que le traitement échoue avec l'adresse e-mail incorrecte \Verb|albert.camus@gmail|. Ce test unitaire vise à vérifier le comportement de la classe \Verb|CheckEmail| lorsqu'elle traite un fichier avec des adresses e-mail incorrectes. Il s'assure que l'exception est générée comme prévu et que le système réagit correctement à cette situation.

\subsubsection{Tests fonctionnels}

La classe \Verb|UploadControllerTest| (Code source~\ref{code:upload-controller-test}, page~\pageref{code:upload-controller-test}) est une classe de tests fonctionnels pour le contrôleur \Verb|UploadController|. Ces tests visent à vérifier différents scénarios d'envoi de données via une requête POST à l'API.

Chaque méthode de test dans la classe correspond à un scénario spécifique. Par exemple, la méthode \Verb|testNoClientAppKey| (Code source~\ref{code:upload-controller-test-no-client-app-key}) vérifie le comportement lorsque la clé \Verb|client_app| est absente dans la requête JSON.

\begin{code}
    \caption{Le classe \Verb{UploadControllerTest}.}
    \inputminted[firstline=10,lastline=19]{php}{code/UploadControllerTest.php}
    \label{code:upload-controller-test-no-client-app-key}
\end{code}

Dans chaque méthode de test, une requête POST est effectuée vers une URL spécifique de l'API avec des données JSON simulées. Par exemple, dans \Verb|testNoClientAppKey|, une requête est envoyée à l'URL \Verb|/api/command/uploads/onboarding/drivers| avec une clé \Verb|parameters| mais sans la clé \Verb|client_app|.

Après avoir envoyé la requête, la réponse est évaluée à l'aide des méthodes d'assertion fournies par Laravel, comme \Verb|assertUnprocessable|. Ces méthodes vérifient si la réponse HTTP est conforme aux attentes.

Ces tests visent à garantir le bon fonctionnement du contrôleur \Verb|UploadController| en simulant différents scénarios d'envoi de données à l'API et en s'assurant que les réponses sont conformes aux attentes. Cela permet de valider le comportement attendu de l'API dans diverses situations.